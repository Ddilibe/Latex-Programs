% !TEX TS-program = pdflatex
% !TEX encoding = UTF-8 Unicode

% This is a simple template for a LaTeX document using the "article" class.
% See "book", "report", "letter" for other types of document.

\documentclass[11pt]{book} % use larger type; default would be 10pt

\usepackage[utf8]{inputenc} % set input encoding (not needed with XeLaTeX)

%%% Examples of Article customizations
% These packages are optional, depending whether you want the features they provide.
% See the LaTeX Companion or other references for full information.

%%% PAGE DIMENSIONS
\usepackage{geometry} % to change the page dimensions
\geometry{a4paper} % or letterpaper (US) or a5paper or....
% \geometry{margin=2in} % for example, change the margins to 2 inches all round
% \geometry{landscape} % set up the page for landscape
%   read geometry.pdf for detailed page layout information

\usepackage{graphicx} % support the \includegraphics command and options

% \usepackage[parfill]{parskip} % Activate to begin paragraphs with an empty line rather than an indent

%%% PACKAGES
\usepackage{booktabs} % for much better looking tables
\usepackage{array} % for better arrays (eg matrices) in maths
\usepackage{paralist} % very flexible & customisable lists (eg. enumerate/itemize, etc.)
\usepackage{verbatim} % adds environment for commenting out blocks of text & for better verbatim
\usepackage{subfig} % make it possible to include more than one captioned figure/table in a single float
% These packages are all incorporated in the memoir class to one degree or another...

%%% HEADERS & FOOTERS
\usepackage{fancyhdr} % This should be set AFTER setting up the page geometry
\pagestyle{fancy} % options: empty , plain , fancy
\renewcommand{\headrulewidth}{0pt} % customise the layout...
\lhead{}\chead{}\rhead{}
\lfoot{}\cfoot{\thepage}\rfoot{}

%%% SECTION TITLE APPEARANCE
\usepackage{sectsty}
\allsectionsfont{\sffamily\mdseries\upshape} % (See the fntguide.pdf for font help)
% (This matches ConTeXt defaults)

%%% ToC (table of contents) APPEARANCE
\usepackage[nottoc,notlof,notlot]{tocbibind} % Put the bibliography in the ToC
\usepackage[titles,subfigure]{tocloft} % Alter the style of the Table of Contents
\renewcommand{\cftsecfont}{\rmfamily\mdseries\upshape}
\renewcommand{\cftsecpagefont}{\rmfamily\mdseries\upshape} % No bold!

%%% END Article customizations

%%% The "real" document content comes below...

\title{Brief Article}
\author{The Author}
%\date{} % Activate to display a given date or no date (if empty),
         % otherwise the current date is printed 

\begin{document}
\maketitle
\tableofcontents\newpage

\chapter{Introduction into Future Fit Career}
\section{Getting started with Future Fit Career}
	The folowing include steps to be successfully while enrolled in the fit future career program.
	\begin{itemize}
		\item Welcome to the fit future community
		\item How to make the most of fit future career navigator
		\item A message to the next generation
		\item Look at the impact you want to have in the world through a future of work lens
		\item Discover new learning opportunities
		\item Start to build your professional network of peers based on shared impact.
		\item Result one - You will learn about the new skills you have to acquire
		\item Result two - You will discover new types of learning pathways
		\item Result three: You will learn about the future ways of working
		\item Result four: You will complete your future fit roadmap
		\item Getting you started witht the fututre fit roadmap
		\item What will you find inside the navigator?
		\item Comment and add to the navigator
		\item The future fit platform: Crowdsourcing, Knowledge and opportunities.
		\item You are already part of the fit future community
		\item Few Recommendation before we start
		\item It is going to be an exciting ride so buckle up.
	\end{itemize}


\section{Stay Ahead of the curve in an accelerating world}
	If you were not embarassed by who you were 12 months ago, you did not learn enough --- Alain De Botton
	\subsection{Two Important lessons from covid-19}
		COVID-19 has taught us two lessons that we should keep in mind when thinking about our next steps:
		\begin{itemize}
			\item  Change can be sudden, unexpected, and outside of our control. 
			\item  We need to be agile learners- always ready to adapt to new realities.
		\end{itemize} 
	\subsection{The Future is faster than you think}
		"Being able to see around the corner of tomorrow and being agile enough to adapt to what's coming, has never been more important". 
		-The Future is Faster than We Think, by Peter Diamandis \& Steven Kotler

		New technologies, new professions, new ways of working \& learning are coming our way at a faster and faster pace. 		This is because technology keeps getting better \& more powerful every 2 years. 
	\subsection{The 4th Induxtrial revolution has picked up its pace}
		What is driving change? Many interlocking trends that are already front and center in our daily lives:
		\begin{itemize}
			\item Climate change
			\item Blacklivesmatter, \#metoo ect... (growing inequality)
			\item Global business 
			\item Mass migration 
			\item Artificial intelligence (emerging \& converging technologies)
			\item Digital transformation
		\end{itemize}
		COVID-19 has accelerated the 4th Industrial Revolution- the convergence of digital, biological, and physical innovation. It's a fusion of advances in artificial intelligence (AI), robotics, the Internet of Things (IoT), 3D printing, genetic engineering, quantum computing, and other technologies. The 4th industrial revolution is affecting us all. In its scale, scope, and complexity, the transformation will be unlike anything humankind has experienced before \& it will bring the most pervasive, disruptive, and lasting change in our lifetime.
	\subsection{10+ technologies are advancing at the same time for the first time in history}
		For the first time in history, there are ten plus new technologies that are maturing all at the same time. Technologies become mature when they deliver value to people and societies, the initial faults have been solved, and they are cheap enough for mass adoption. 
		The technologies of the 4th industrial revolution have two distinctive features: 
		\begin{itemize}
			\item  First, they are stacking and recombining with each other to create new applications that are transforming markets and jobs in ways never seen before
			\item  Second, they advance exponentially. Let's see what it means to move following an exponential curve. The ten technologies include:
			\begin{itemize}
				\item Quantum Computing
				\item Artificial Intelligence
				\item Robotics
				\item Nanotechnology
				\item Biotechnology
				\item 5g Networks
				\item Sensors and Internet of Things
				\item 3D printing
				\item Augmented and Virtual Reality
				\item Blockchain
			\end{itemize}
		\end{itemize}
	\subsection{The exponential Effect: Technologies double in speed at every step}
		The technologies are advancing exponentially because they follow Moore's Law. 
		In 1965, Gordon Moore, the co-founder of Intel Corporation, observed that: "every 18 months, the power of computer circuits and components doubles, while their size and price shrink". 
		If you have a hard time picturing how fast is "exponential," look at the difference between taking 30 linear steps (always of the same length) and 30 exponential steps (that double in length at every step).
	\subsection{A simple explanantion of Moore\rq{}s law}
		Consumer technology generally progresses according to three criteria: size, speed, and convenience. It must be small, fast, and easy in order to be adopted....and Moore's law makes it so. 
		Why is this important? Moore's law, improves the cost, performance and power consumption with every new generation of technology. So if you think the new technologies fall short \& can't do seamlessly what you want them to do, just wait a year or two. They will exponentially get better.
	\subsection{Takeaway 1}
		The world around you is changing at an increasingly RAPID pace. 10+ technologies are growing exponentially and converging to radically change the way we work and learn \& it's happening now.
		This is the slowest pace of change you will experience. If you do not want to be caught unprepared, you will have to continuously anticipate and adapt to the new realities. The question is: How do you do it?
	\subsection{Spend more time with your future}
		Why should you care about a future in acceleration? Because it is wiser to decide your NEXT STEP in light of the changes that the 4th industrial revolution is bringing, rather than focusing solely on the career paths trending in the past decade. Many of them are at risk of being radically reshaped. According to the World Economic Forum, by 2022, no less than 54\% of all employees will require significant reskilling \& upskilling.
		\newline In the OCED 2019 PISA survey, 47\% of 15-year-old boys and 53\% of 15-year-old girls from 41 countries and economies expressed interest in just ten career types. These numbers raise questions about the extent to which students are aware of the new roles, future jobs, and new opportunities.
		What about you? Are you aware of the new job opportunities and learning paths that will be opening up in the coming decade? Do you ever pause and think about what the future will be? Are you "Future Literate"? If not, no worries, we'll get you covered. 
	\subsection{Being \lq\lq{}Future Literate\rq\rq{}}
		Being “Future Literate” is the skill that allows you to follow trends, read signals, better imagine and make sense of the opportunities the future will bring. 
		Getting “Future Literate,” doesn’t mean you will be able to predict the future – as if it were predetermined – instead, it means, you consider the future as something that you can create and not as something already decided. 
		
		Jane McGonigal teaches a class at Stanford University, How to think like a Futurist and become “Future Literate” (available for free enrollment on Coursera). She says:
		
		“Thinking about the far-off future is more than an exercise in intellectual curiosity. It is a practical skill that new research reveals has a direct neurological link to greater creativity, empathy, and optimism. In other words, futurist thinking gives you the ability to create change in your own life and the world around you, today.”  
		
	\subsection{ What do you do to become a future literate?}
		Here are three simple steps that you can take to become "Future Literate".
		\begin{itemize}
			\item STEP 1:  Make some time:

				Dedicate some of your daily time to investigate \& imagine how your life will evolve in the next five years. Do it on your own or even better with some friends. 

			\item STEP 2:  Look for signals:
			
				Look for signals of change and mega-trends that will impact your future choices as a human being, as a student, and as a professional.

			\item STEP 3:  Imagine yourself in the future:

				Imagine and visualize scenarios where you put yourself in the future, five to ten years from now, functioning in a world dominated by the trends and change you have thought about in step 2.
		\end{itemize}		
				
	\subsection{Takeaway 2}
		In an accelerating world, skills, degrees, and jobs will become obsolete in a much shorter time: so what sounds like a good degree or career right now might be outdated sooner than you think. 
		
		Being "Future Literate" -making sense of your choices in the context of the changes that are likely to happen five years from now- will help you take the right decision for your NEXT STEP.
		
		In the last part of this section, we will share several resources that can become part of your daily personalized "newsfeed" and help you build your "Future Literacy" skills. 
		
\chapter{Discover the big impact goal in the future of work}
	\section{The occupational Identity}
		Throughout your life, you have probably been asked many times, “What do you want to be when you grow up?” 

		Well, that question is becoming irrelevant for students these days. 

		The Foundation for Young Australians predicts that your generation will hold 17 jobs across 5 different industries in the course of your “career” journey. While the numbers aren't set in stone, change is the new normal in our careers. 

		If the world of work is changing so rapidly, and many jobs will be automated or disappear in the near future, does it make sense to tie your vision and your “next step” choices to a single occupational identity?  (for example: “I want to be a manager, a lawyer, a doctor, a singer”)

		Not anymore: it doesn’t. We believe occupational identities will soon be a thing of the past.

		Watch, Emilie Walking, describe the “multipotentialites” of people and argue that, in life, you can be many things at the same time.
	\section{Typing your identity and your growth to a purpose or a mission}
		Most Students don\rq{}t know what they want to do after graduation. We surveyed 333 students from eight different higher education institutions:
		\begin{itemize}
			\item 50\% of students don;t have high clarity about their future career
			\begin{itemize}
				\item 68\% of students don\rq{}t understand what skilld are needed to start their career
				\item 60\% of students have high confidence in getting a job that align with their goals
				\item 50\% of students know how to search for a specific job
			\end{itemize}
		\end{itemize}
		In times of continuous change, we need to stop thinking of ourselves as going to university to become "a profession". Instead, we need to start looking at our lives as: 

		"Beta versions", "Prototypes of ourselves" in continual states of change and improvement guided by a purpose bigger than us.
(Kate O'Keeffe, Cisco Hyper Innovation Living Labs) 
	
		As technological and social change push us to take on several different jobs within our lifetime, we must connect to our "Why", to help give us direction and meaning to our lives. In other words, you should not define yourself by what you do (because rapid change due to AI \& automation will change that often), but start defining yourself based on why you do it: your purpose. 
		\subsection{Starting with your \lq{}Why\rq{}}
			Success comes when we wake up every day in that never-ending pursuit of WHY we do WHAT we do. (Simon Sinek)
	\section{Your big impact Goal}
		At Future Fit, we believe that when future-proofing your NEXT STEP after graduation, the decision should be guided by the Big Impact Goal you want to pursue in your life. 

		What do we mean by a Big Impact Goal? 
		Our generation has a big responsibility in putting society and the planet on a dramatically different course. We believe that, when imagining your future, you should think about the contribution, the impact you want to make in solving some of the urgent challenges of our times. 

		This Big impact Goal will act as a north star to help you decide your next-steps. This is a new way of looking at a career \& taking decisions about you next steps in a rapidly changing world.

		Pick a Big Impact Goal (an urgent global challenge) and then curate your learning and work experience around helping solve that problem. Using your Impact goal as your north star,  will help you navigate uncertainty and rapid change. You will nurture your curiosity, purpose and passion; stay motivated when change is happening all around you; and you will set yourself on a path of fulfillment and lifelong learning, working with other like-minded peers on something bigger than yourself.
		
		The Big Impact Goal, the first part of your Future Fit Roadmap, is the foundation on which you will base your NEXT STEP. 

Your Big Impact Goal helps you answer three main questions: \newline
1. What is the big challenge that I want to tackle?\newline 
2. What is the change that I want to see?\newline
3. How do I want to contribute to that change?

		\section{The six global challenges}
			Dig deeper on the challenges that interests you the most. The six challenges are
			\begin{itemize}
				\item CHALLENGE 1: CLIMATE CHANGE
				How might we accelerate the transition to a zero-carbon economy? 

				\item CHALLENGE 2: HEALTHCARE
				How might we make healthcare affordable, effective, and personalized for all? 
				
				\item CHALLENGE 3: EDUCATION
				How might we reinvent education for universal lifelong learning and employability? 
				
				\item CHALLENGE 4: HUMANE TECHNOLOGIES
				How might we develop A.I. and Tech Solutions that empower every human?
				
				\item CHALLENGE 5: SOCIAL INEQUALITIES
				How might we support the growth of equitable, inclusive, and just societies?
				
				\item CHALLENGE 6: SPACE EXPLORATION
				How might we further space exploration to make life on earth more sustainable? 
			\end{itemize}
	\section{Humane Technologies Challenges}
		Fantastic that you are interested in a career in Technology \& you want to make sure that it is human-centered. That it benefits all. This will be one of the greatest challenges in the coming decades. 
		
		Let's start by learning more about the Challenge problem space of Responsible, Ethical Humane technology.
		
		The Challenge
		For over two decades, the world has looked at the Digital Era with optimism and the belief that the internet, personal computers, and smartphones would only improve our lives. 
		
		However, in recent years we have witnessed some dangerous side-effects of the new technologies on our societies: the rise of fake news, breaches of privacy, social divisions, racial bias, and rising disparities between tech billionaires and the rest of society.
		
		The increasingly pervasive use of technology in our everyday lives has triggered a debate on how new and disruptive technologies – such as artificial intelligence (AI), robotics, 3D printing, internet of things (IoT), 5G, blockchain, quantum computing, autonomous vehicles, biotechnology, and nanotechnology – should be managed and governed so that the technology works for all.
		
		It has us asking: Does information technology (IT) solve problems and make our lives easier, allowing us to do more with less? Or does it introduce additional complexity to our lives, isolate us from each other, threaten privacy, destroy jobs, and generate an array of other harms?
		
		As government struggles to keep up with the unprecedented speed and scale of technological change, there is a growing distrust toward “Big Tech” companies and generalized opposition to technological innovation \& support for policies that are expressly designed to inhibit it. That is deeply problematic for future progress, prosperity, and competitiveness.This important “distrust” challenge requires the engagement of stakeholders across the whole technology value chain, from the initial design to the sale of technology and its ultimate end use. Additional urgency is driven by the effects of the Coronavirus pandemic on society and the increased need to responsibly use potentially life-saving technology.
		- (World Economic Forum)
		
		"There’s a real danger that without proper training on data evaluation and spotting the potential for bias in data, vulnerable groups in society could be harmed or have their rights impinged. AI also has intersectional implications on criminal and racial justice, immigration, healthcare, gender equity, and current social movements."
		-Shalini Kantayya, Director of Coded Bias
		
		Current policies and ethical guidelines for AI technology are lagging behind the progress artificial intelligence has made in many industries. By \& large, society remains ill-informed of the ethical complexities that AI technology can introduce. 
		
		The widespread adoption of Artificial Intelligence and emerging technologies with the potential to infringe on our liberties and rights as human beings is a problem that is increasingly disconcerting for societies all over the world.
		
		Read at this LINK about China’s Social Credit Experiment.
		
		Learn more HERE about Facebook’s Cambridge Analytica scandal, in which a research firm used the social media giant’s advertising platform to send personalized content to millions of users to affect their vote in the 2016 U.S. presidential elections.
		
		Technology, like global wealth and power, is out of balance - a few people control and call the shots which affect a million lives. It is important to acknowledge that and understand all the facets of technology, identifying gaps and repairing them with collaborative efforts. It is important to understand that this rapidly evolving tech landscape, and the foundation for the future of responsible technology has to be rooted in democratic, anti-racist, anti-colonial, equitable and inclusive principles. (-All Tech is Human).
		
		Let's watch a video, the trailer of "The Social Dilemma" that will help us understand the problem better. (If it is possible you should watch the whole documentary "The Social Dilemma" on Netfllix)
	\section{The Vision: Develop Tech Solutions Empowering Every Human}
		Now you know some of the challenges in the debate around technology, let's look at the Vision, the future we want to live in when it comes to humane technology and what we have to do to get there.
		 
		What is Responsible, Ethical \& Humane TECH?
		\paragraph{}The phrase "Humane Tech" or “Responsible Tech”or “Ethical Tech” refers to the multi-disciplinary field that aims to better align the development and deployment of digital technologies with individual and societal values and expectations. 
		
		The "All Tech Is Human" community define it as: the field focused on reducing the harms of technology, diversifying the tech pipeline of people who work in technology, and ensuring that technology is aligned with the public interest, that it is inclusive and not doing harm.
		 
		Humane tech addresses issues like: algorithmic bias, potential job losses through advancing technology, data privacy, the role of tech platforms in moderating speech, the ethics regarding autonomous vehicles and the ethical design of products/apps.
		 
		Aligning technology with the “public interest” requires a better connection between the communities that develop digital technologies and the communities that these technologies impact, with the goal of reducing harm and increasing benefits to all.
		 
		Similar to the sustainability movement that seeks to better align business behavior with the interests of individuals and society (thereby reducing harms),the Humane Tech field aims to better align digital technologies with our civil liberties, notions of fairness, and democracy at large. Our organization advocates taking a proactive, systemic approach towards ensuring that the development and deployment of technology is co-created and aligned with the public human interest. (-All Tech is Human)
		 
		The overall vision is to facilitate AI systems and technologies that enhance human capabilities and empower individuals and society as a whole while respecting human autonomy and self-determination.
		(-Humane AI Net).
		 
		We need the people \& organizations who are building AI dedicating their focus and energy to the development of more "humane technologies": technologies intentionally designed to support our well-being, our democracies, and our shared knowledge. 
		 
		Humane technologies are designed to be socially responsible, providing benefits for users and society while minimizing the adverse effects that we have seen playing out in recent times, such as:
		\begin{itemize} 
			\item information overload or disinformation; 
			\item breaches of individual privacy, 
			\item discrimination due to algorithmic bias, or lack of transparency in the way AI systems base their decisions. 
		\end{itemize}
		 
		At the core of "humane technologies" is the idea that ethics needs to become simply a part of how AI and tech applications are made and used, rather than an add-on or afterthought. (Jessica Whittlestone).
		 
		“Responsible Tech sits at the heart of ensuring an inclusive, just and equitable society- and there's a need for those of us in the social sciences and humanities, from across diverse communities, to advance what is being developed in the right direction and represent society more holistically ” - 
		Gullnaz Baig, Director of Trust and Safety, Aaqua
	\section{Learn more about humane technologies}
		

	
\end{document}
