\documentclass{book}
\usepackage{listings}
\title{LaTex Docoument}
\author{Wikipedia, \\collected by firuza }
\date{\today}
\begin{document}
	\maketitle
	\tableofcontents
	\chapter{Overview \LaTeX}
	
	\section{Overall Description}
	
	\section{Introduction}
	
	\paragraph{}
	Dhansak is a popular Indian dish, originating among the Parsi Zoroastrian community.[1] It combines combines elements of persian and gujarati cuisine. Dhansak is made cooking mutton or goatmeat with a mixture of lentils and vegetables. This is served with caramelised brown rice which is cooked in caramel water to give it a typical taste and colour. The dal cooked with mutton and vegetables served with brown rice, altogether is called dhansak.
	
	\lstset{
		basicstyle=\small,
		keywordstyle = \color{black}\bfseries\underbar,
	}
	\begin{lstlisting}
		#include<stdio.h>
		#include "main.h"
		/**
		   * main - Check the code
		   * 
		   * Return: Always 0
		   */
		int main(void){
			int f;
			
			f = 2;
			
			if (f <= 1)
			{
				return (f * 2);
			}
		}
	\end{lstlisting}
	\subsection{Purpose}
	
	\subsection{Scope}
\end{document}