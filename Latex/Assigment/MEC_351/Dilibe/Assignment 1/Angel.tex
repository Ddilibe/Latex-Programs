\documentclass{article}
\title{MEC 331  Assignment}
\author{Emersin Miguel  C. REG. NO: 2018/241792 }
\begin{document}
\maketitle

\newpage

\begin{center}\title{\section*{\textbf{{Assignment}}}}\end{center}
\section*{\textbf{Question 1.7:}}


\begin{center} Solution \end{center}

\[\sum{}F_{(x)}=0,  \sum{}F_{(y)}=0 \]
\[F_{BD} - F_{CE} = -20\]
$\sum{}M_{(c)} = 0$
$(0.040)F_{BD} + (0.025 + 0.040)(20 \times 10^{3}) = 0 $
\[ F_{BD} = 32.5\times 10^{3}.  \]  \begin{center}This force  is a Tension force\end{center}

$\sum{}M_{(b)} = 0$
$-(0.040)F_{CE} - (0.025 )(20 \times 10^{3}) = 0 $
\[ F_{BD} = -12.5\times 10^{3}. \] This force is a compression force



The area of link in tension  $A = (0.008)(0.036 - 0.016) = 150 \times 10 ^{-5}$

For two parallel links, $A_{t} = 320 \times 10 ^ {-6} $

 the tensile stress in link AB


\textbf{Link BD}
\[\delta_{BD} = \frac{F_{BD}}{A} = \frac{32.5 \times 10^{3}}{320 \times 10^{-4}} = {101.56 \times 10^{6}} or 101.6MPa\]

The area of link in Compression  $A_{t}= (0.008)(0.036 ) = 288 \times 10 ^{-6}$

For two parallel links, $A_{c} = 576 \times 10 ^ {-6} $


the compressing stress in link CE

\textbf{Link CE}
\[\delta_{CE} = \frac{F_{CE}}{A} = \frac{-12.5 \times 10^{3}}{320 \times 10^{-4}} = {-21.7 \times 10^{6}} or 21.7MPa\]



\section*{\textbf{Question 1.26:}}

\begin{center} Solution \end{center}
\[A = bt\]
where b = 50mm and t = 6mm
\[P = -\delta \times A = -(-140 \times 10^{6})(50 \times 10^{-3} \times 6 \times 10^{-3}) = 42KN\]
Pin: $ T_{p} = \frac{P}{A_{p}} $ and $A_{p} = \frac{\pi}{4}d^{2}$
\begin{itemize}
\item The diameter,d of the pins
\[d = \sqrt{\frac{4\times A_{p}}{\pi}} = \sqrt{\frac{4\times P}{\pi\times\tau_{p}}} = \sqrt{\frac{4\times 42 \times 10^{3}}{\pi \times 80 \times 10^{6}}} = 0.025m\]
\item The average Bearing stress in the link
\[\delta_{b} = \frac{P}{d\times t} = \frac{42\times10^{3}}{0.025\times0.006} = 280MPa\]
\end{itemize}




\section*{\textbf{Question 1.11:}}

\begin{center} Solution \end{center}

% This is the answer section

\[\sum{}M_{a} = 0, D_{x} = (45 + 30)(480) = 0\]
\[ D_{X} = 900lb \]
\begin{center} Using Member DEF as a free body \end{center}
\[\sum{}F = 0\]
\[\frac{3}{5}D_{y} - \frac{4}{5}D_{x} = 0\]
\[D_{y} = \frac{4}{5}D_{x} = 1200lb\]
Now Taking Moment at different points of the beam DEF,
\[\sum_{} M_{f} = 0 \]   \[-(30)(\frac{4}{5}F_{BE}) - (30+15)D_{y} = 0\]    \[F_{BE} = -2250lb \]
\[\sum_{} M_{E} = 0 \]   \[-(30)(\frac{4}{5}F_{CE}) - (15)D_{y} = 0\]    \[F_{BE} = 750lb \]


Area of member BE is A = $2in \times 4in = 8in^{2}$

Area of the cross section which occurs at the pin is $A_{min} = (2)(4.0-0.5)=7.0in^{2}$

\[\delta_{BE} = \frac{f_{BE}}{A} = \frac{-2250}{8}= -281psi\]
\[\delta_{CE} = \frac{f_{CE}}{A_{min}} = \frac{750}{7.0}= 107.1psi\]





\section*{\textbf{Question 1.15:}}

\begin{center} Solution \end{center}

\[ \delta = \frac{p}{t \times L}\]
\[ \delta = \frac{8\times 10^{3}}{15 \times 10^{-3} \times 90 \times 10^{-3}} = 5.93MPa\]




\section*{\textbf{Question 1.16:}}

 \begin{center} Solution\end{center}

\[F = 12KN = 12 \times 10^{3}N\]
The average shear stress in the glue is 120 psi $\tau = 120psi$
\[\tau = \frac{F}{A} \]
\[A = \frac{F}{\tau} = \frac{5800}{120} = 48.333in^{2}\]
let L = length of glue Area and W =  width = 4in
\[A= L \times W\]
\[l = \frac{A}{W} = \frac{48.333}{4} = 12.083in\]
 L = $2L +$ gap = $(2\times 12.08)+ \frac{1}{4}$ = 24.42in

\section*{\textbf{Question 1.17:}}

\begin{center} Solution \end{center}
For Steel, 
\[A_{1} = \pi \times d\times t= \pi (0.6)(0.4) = 0.7540in^{2}\]
\[\tau_{1} = \frac{P}{A_{1}}\]
\[P = A_{1}\times\tau_{1} = (0.7540)(18) = 13.57kips\]
for Aluminum 
\[A_{2} = \pi\times d\times t = \pi(1.6)(0.25) = 1.2566in^{2}\]
\[\tau_{2} = \frac{P}{A_{2}}\]
\[P= \tau_{2} \times A_{2} = (1.2566)(10)= 12.57kips\]
Limiting value of P is the Smaller Value: P = 12.57kips



\section*{\textbf{Question 1.26:}}

\begin{center} Solution \end{center}
\[A = bt\]
where b = 50mm and t = 6mm
\[P = -\delta \times A = -(-140 \times 10^{6})(50 \times 10^{-3} \times 6 \times 10^{-3}) = 42KN\]
Pin: $ T_{p} = \frac{P}{A_{p}} $ and $A_{p} = \frac{\pi}{4}d^{2}$
\begin{itemize}
\item The diameter,d of the pins
\[d = \sqrt{\frac{4\times A_{p}}{\pi}} = \sqrt{\frac{4\times P}{\pi\times\tau_{p}}} = \sqrt{\frac{4\times 42 \times 10^{3}}{\pi \times 80 \times 10^{6}}} = 0.025m\]
\item The average Bearing stress in the link
\[\delta_{b} = \frac{P}{d\times t} = \frac{42\times10^{3}}{0.025\times0.006} = 280MPa\]
\end{itemize}


\section*{\textbf{Question 1.18:}}


\begin{center} Solution \end{center}
Seven surfaces carry the total load P = 1200lb

Area \[A = (7)(\frac{7}{8})d = \frac{49}{8}d\]
\[\tau = \frac{P}{A}\]
\[A = \frac{P}{\tau}\]
\[\frac{49}{8}d = \frac{1200}{120} \]
Therefore, d = 1.683in




\section*{\textbf{Question 1.23:}}
 
\begin{center} Solution\end{center}
\begin{itemize}
\item Maximun Average Normal Stress In the Wood
\[ A_{net} = 1\times(4-\frac{5}{8}) = 3.375in^{2}\]
\[\delta = \frac{P}{A_{net}} = \frac{1500}{3.375} = 444.44psi\]
\item  Distance B from which the average shaering stress is 100psi onn the surface indicated by the dashed lines
\[\tau = \frac{P}{A} = \frac{P}{2\times b\times t}\]
\begin{center}Makinig B the subject of the formualr \end{center}
\[b = \frac{P}{2\times t\times\tau} = \frac{1500}{2\times 1\times100} = 7.5in\]
\item Average Bearing Stress On the Wood
\[\delta = \frac{P}{A_{b}} = \frac{P}{d\times t}=\frac{1500}{1\times\frac{5}{8}} =2400psi\]
\end{itemize}






\end{document}
