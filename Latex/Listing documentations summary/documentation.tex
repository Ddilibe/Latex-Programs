\documentclass{article}
\usepackage{listings}
\title{A summariized use of the listing package}
\author{Dilibe}
\begin{document}
	\maketitle
	\lstset{,
	basicstyle=\small, % print whole listing small
	keywordstyle=\color{blue}\bfseries\underbar,% underlined bold black keywords
	identifierstyle=, % nothing happens
	commentstyle=\color{white}, % white comments
	stringstyle=\ttfamily, % typewriter type for strings
	showstringspaces=false,
	numbers=left, numberstyle=\tiny, stepnumber=2,numbersep=5pt}
	\tableofcontents
	\section{Introduction}
	To get started with the listing package, it is loaded with the command\textbackslash usepackage[(options)]\{listing\}
		there are three types of codes which are surpported namely: code snippets, code segments and listing pf stand alone files.
	\subsection{code snippets}
		this code is used if the codde you are willing to write is put inside a sentence.
		it goes with the code \textbackslash lstinline{the code you wan to write}. the curly barckets are delimiters and they can be replaced with any other characyers that is not in the code.
	\subsection{Displayed Code}
		This type set encloses it and displayes it on its own as a picture. The synatax goes like this
		\begin{lstlisting}
		\\begin{lstlisting}
			\**
			  * this is real
			  */
			for i :== maxint to 0 do
			begin
				{do  nothing}
			end:
			
			Write(Case_sensitive)
			write (Pascal_keywords)
		\/end{lstlisting}
		\end{lstlisting}
	\subsection{Stand alone files}
		This surpported form means to add external code files into the latex program. you have to correctly define the file path. this is the syntax: \textbackslash lstinputlisting{listing.sty}.
		\lstinputlisting{go.c}
	These are all the ways that the source file of a code in other programming languages can be inputed.
	
	\section{How to edit the appearances of the code in latex}
		Keywords are used to to change things in thr inouted codes like font size, the color, and a lot of things.
		Sadly, this design only works with displayed codes. and not inline or stand alone files.
		To start ti design, you use the \textbackslash lstset, then opem your curly brackets, ass the following
		\\basicstyle = This is used to chabge the font size
		\\keywordstyle - This is used to highlight keywords  used in the code. it takes two argument a command style and an underbar command. The sype include \textbackslash ttfamily and the \textbackslash bfseries then the \textbackslash underscore.
		\\FLoating: By floating, the codes will appear on top of the page. 
	\subsection{Other features that appear include:}
		\begin{lstlisting}
			if (i<=0) then i := 1;
			if (i>=0) then i := 0;
			if (i<>0) then i := 0;
		\end{lstlisting}
		
	\section{Alternatives to the listing mechanism in \LaTeX }
	\paragraph{}
		This program also have other competitors. Some of them do the job well. Some are independent of \LaTeX while others come with a separate program plus a \LaTeX package.These packages include:
		\\
	\section{\LaTeX commands and their explaintion}
		\textbackslash{lstloadlanguages}\{<comma separated list of languages>\}
\end{document}